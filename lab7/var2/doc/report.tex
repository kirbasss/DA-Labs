\documentclass[12pt]{article}

\usepackage{fullpage}
\usepackage{multicol,multirow}
\usepackage{tabularx}
\usepackage{ulem}
\usepackage[utf8]{inputenc}
\usepackage[russian]{babel}

\begin{document}

\section*{Лабораторная работа №\,7 по курсу дискрeтного анализа: Жадные алгоритмы} 

Выполнил студент группы 08-303 МАИ \textit{Арусланов Кирилл}. 

\subsection*{Условие}

Даны $N$ отрезков $[L_i, R_i]$ на координатной прямой. Необходимо выбрать минимальное количество отрезков, которые полностью покрывают интервал $[0, M]$.  

\subsection*{Метод решения}

Для решения задачи применяется \textbf{жадный алгоритм}.  
Пусть текущая правая граница покрытия равна \texttt{cur}. Среди всех отрезков, начинающихся не правее \texttt{cur}, выбирается тот, который имеет наибольший конец \texttt{R}. Этот отрезок добавляется в решение, и правая граница \texttt{cur} сдвигается до \texttt{R}.  
Процесс повторяется, пока \texttt{cur < M}. Если на каком-то шаге подходящих отрезков не находится, покрытие невозможно.

Сложность алгоритма:  
\[
O(N \log N)
\]
— на сортировку по левым концам,  
\[
O(N)
\]
— на один проход по массиву.  
Итого — асимптотика $O(N \log N)$.

\subsection*{Описание программы}

Программа реализована на языке C++.

\begin{itemize}
    \item \textbf{struct Seg} — структура для хранения отрезка $(L, R)$ и исходного индекса.
    \item Основной алгоритм:
    \begin{enumerate}
        \item Чтение $N$ и всех отрезков.
        \item Сортировка отрезков по $L$ с помощью лямбда-функции.
        \item Итеративный жадный выбор отрезков, расширяющих покрытие.
        \item Вывод результата в исходном порядке появления.
    \end{enumerate}
\end{itemize}

\subsection*{Дневник отладки}

Проблем при разработке не возникало, программа прошла чеккер с первой попытки.

\subsection*{Тест производительности}

Были проведены эксперименты на наборах данных размером $N = 1000$, $5000$, $10000$, $20000$, $50000$.  
В таблице приведены усреднённые времена выполнения по 5 испытаниям (в миллисекундах):

\begin{center}
\begin{tabular}{|c|c|c|}
\hline
$N$ & $M$ & Среднее время работы, мс \\
\hline
$1000$ & $1000$ & 0.08 \\
$5000$ & $5000$ & 0.43 \\
$10000$ & $10000$ & 0.79 \\
$20000$ & $20000$ & 2.7 \\
$50000$ & $50000$ & 10.9 \\
\hline
\end{tabular}
\end{center}

Рост времени работы соответствует зависимости $O(N \log N)$: при увеличении $N$ в 5 раз время увеличивается примерно в 10–12 раз, что согласуется с теоретической сложностью сортировки.

\subsection*{Недочёты}

Программа предполагает, что входные данные корректны и не содержит проверки на отрицательные значения координат или пустые отрезки.

\subsection*{Выводы}

Реализован жадный алгоритм минимального покрытия интервала.  
Подход эффективен и оптимален для данной задачи: на каждом шаге выбирается локально лучший отрезок, что приводит к глобально оптимальному решению.  
Алгоритм имеет временную сложность $O(N \log N)$ и хорошо масштабируется при увеличении объёма входных данных.

Данный метод применяется в задачах оптимального планирования, интервалов времени, покрытия диапазонов, маршрутизации и обработке событий.

\end{document}